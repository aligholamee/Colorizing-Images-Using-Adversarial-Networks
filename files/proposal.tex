\documentclass{article}

\begin{document}

\centerline{\sc \large Project Proposal}
\vspace{.5pc}
\centerline{\sc Team Name: Just Add Another Layer}
\centerline{\sc Team Members: Cameron Fabbri, Jahidul Islam}
\centerline{\sc 2/5/2017}
\vspace{2pc}

\centerline{\sc \large Generative Adversarial Networks for Automatic Image Colorization }

\section{Introduction}
There exists a large amount of photographs and videos, mainly antique, that lack color.
Providing color to these images provides a modern view to each scene. For a human,
the task of colorizing these black and white photos leaves open room for imagination. While
some objects commonly hold the same color (e.g grass is \textit{usually} green), many are
left up for interpretation. For example, given a black and white photo of someone wearing a dark
colored shirt, it would be very difficult or near impossible to tell whether that shirt was
blue or green. The loss of information and the fact that there is not one correct answer,
automatic colorization is an ill-posed problem.

\section{Approach}
Recent advances in deep learning and big data provide us with a good starting point for tackling this
problem. We propose to leverage two specific areas to solve this problem. The first is deep Convolutional
Neural Networks (CNNs). Literature has shown[1] that these can be used to provide a \textit{plausible}
colorization of a black and white photo. The second is Generative Adversarial Networks (GANs). Recently,
GANs have shown very promising results for generating data. Energy-Based GANs[2] have been shown to
offer greater stability during training, as well as generate higher resolution images.
We believe the combination of these two methods should improve on automatic colorization.

\section{Objectives}
Given the nature of current research in deep learning, there are many variations in which we
can try. We have four main tasks that we will be exploring that will ultimately converge.

\vspace{2pt}
(1) Implement the state of the art[1] for image colorization. This will provide
\indent us not only with a baseline, but given the impressive results, it would make
\indent sense to use this architecture as the architecture for the generator in our
\indent GANs.

\vspace{3pt}
(2) Implement Deep Convolutional Generative Adversarial Networks \newline
\indent (DCGANs)[3]. 

\vspace{3pt}
(3) Implement Energy-Based Generative Adversarial Networks (EBGANs).[2]

\vspace{3pt}
(4) Explore methods to pretrain the model used in 1 and fine tune it as a
\indent generator in 2 and 3.

\vspace{4pt}

\noindent Both (2) and (3) will be implemented as they are in their respective papers for generating
images. After we are certain the results are valid (i.e the implementation is correct), we will alter
them for the main problem at hand. After each piece has been implemented, it is straight-forward to
replace the generator in 2 and 3 with the architecture from 1. While much of the work will overlap as the
project progresses, our current plan for task delegation is as follows. \newline

\noindent Jahidul: 1, 4 \newline
\noindent Cameron: 2, 3



\section{References}
[1] Zhang, Richard, Phillip Isola, and Alexei A. Efros. "Colorful image colorization." 
European Conference on Computer Vision. Springer International Publishing, 2016.

[2] J. Zhao, M. Mathieu, and Y. LeCun.  Energy-based Generative Adversarial 
Network. ArXiv e-prints, September 2016.\newline

[3] Radford, Alec, Luke Metz, and Soumith Chintala. "Unsupervised representation learning with deep
convolutional generative adversarial networks." arXiv preprint arXiv:1511.06434 (2015).




\end{document}
